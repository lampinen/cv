% LaTeX file for resume 
% This file uses the resume document class (res.cls)

\documentclass[margin]{res} 
% the margin option causes section titles to appear to the left of body text 
\textwidth=5.2in % increase textwidth to get smaller right margin
%\usepackage{helvetica} % uses helvetica postscript font (download helvetica.sty)
%\usepackage{newcent}   % uses new century schoolbook postscript font 
\usepackage{url}
\begin{document} 
 
\name{Andrew Kyle Lampinen\\[14pt]}
 
\address{{\bf Address} \\ Department of Psychology \\ Stanford University \\ 450 Serra Mall \\ Stanford, CA 94305}
\address{{\bf Contact Information} \\ lampinen@stanford.edu \\\url{web.stanford.edu/~lampinen}}
\begin{resume} 
\section{Education} 
{\bf Ph.D. Psychology,} Stanford University, Fall 2015-Present
\begin{itemize} \itemsep -2pt \item Advisor: James L. McClelland. \item Area: Cognitive. \item Center for Mind, Brain, Computation, and Technology Trainee (Co-Mentor: Surya Ganguli). \item Minor in Computer Science.\end{itemize}
{\bf B.A. Mathematics, Physics,} UC Berkeley, May 2014 \begin{itemize} \itemsep -2pt \item Highest honors in mathematics, high distinction in general scholarship. \item GPA: 4.0 Math, 3.9 Physics, 3.9 cumulative. \item Study Abroad Internship, A*STAR IHPC Singapore, Summer 2012. (See Research Experience.) \end{itemize}
\vspace{1pt}\section{Honors} \vspace{-15pt} \rule{\textwidth}{0.5pt} \\[3pt]
Ric Weiland Graduate Fellowship in the Humanities and Sciences, Fall 2018-Present \\
National Science Foundation Graduate Research Fellowship, 2015-2018 \\
Percy Lionel Davis Award for Excellence in Scholarship in Mathematics, May 2014 \\ 
Berkeley Physics Olsen Scholar 2013-2014 \\
Berkeley Letters \& Science Dean's List 2012-2014\\
Berkeley Physics Undergraduate Research Scholar, Spring \& Fall 2012
%\vspace{0pt}
\vspace{1pt}\section{Research Experience} \vspace{-15pt} \rule{\textwidth}{0.5pt} \\[3pt]
{\bf PhD Candidate,} Stanford University Department of Psychology, August 2015 - Present 
\begin{itemize} \itemsep -2pt
  \item Empirical and theoretical investigations of generalization, transfer, and abstraction in deep learning models. 
  \item Research on reducing the quantity of data required to train a deep learning system, including transfer, memory, and curriculum learning.
  \item Research on zero-shot performance of new tasks by transforming task representations.
  \item Experiments to investigate the effects of presentations of concepts on learning of related concepts in mathematical cognition.
\end{itemize}\vspace{-8pt}
{\bf PhD Intern,} DeepMind, May 2019 - September 2019
\begin{itemize} \itemsep -2pt
  \item Explored automated curriculum generation for goal-conditioned reinforcement learning.
  \item Explored generalization in reinforcement learning.
\end{itemize}\vspace{-8pt}
{\bf PhD Software Engineering Intern,} Google Brain, June 2017 - September 2017 
\begin{itemize} \itemsep -2pt
  \item Designed and developed a system for using low-quality data from human interactions to improve an adversarially trained image generative model. 
  \item Contributed gradients to TensorFlow image resizing ops.
\end{itemize}\vspace{-8pt}
{\bf Associate Professional Staff I,} Johns Hopkins University Applied Physics Laboratory, June 2014 - July 2015 
\begin{itemize} \itemsep -2pt
 \item Worked on image classification using convolutional neural networks. \item Developed models and simulations of sensor systems, shipping and transportation, and autoimmune diseases. \item Devised metrics for assessing sensors. \item Worked on methods for identifying malicious software based on its behavior. \end{itemize}\vspace{-8pt}
 {\bf Student Research Associate,} Lawrence Berkeley National Laboratories, January - May 2012 \& August - December 2012
\begin{itemize} \itemsep -2pt
  \item Developed simulations of processes in nuclear physics. \item Engineered software and hardware for efficiently collecting \& analyzing data. \end{itemize}\vspace{-8pt}
{\bf Summer Research Intern,} A*STAR Institute of High Performance Computing, Singapore, June - August 2012
\begin{itemize} \itemsep -2pt
  \item Wrote and adapted simulations of crystallization processes in super-cooled metals. \item Developed software for analyzing and visualizing the structure of crystals. \end{itemize}\vspace{-8pt}
{\bf Research Assistant,} UC Davis Plant Sciences, June - August 2011
 \begin{itemize} \itemsep -2pt
  \item Developed procedures and software for testing the physical attributes of fruit. \end{itemize}

\vspace{1pt}\section{Publications and Proceedings} \vspace{-15pt} \rule{\textwidth}{0.5pt} \\[3pt]
\textbf{Andrew K. Lampinen} and James L. McClelland, (2019), {``Zero-shot task adaptation by homoiconic meta-mapping,''} \textit{Learning Transferable Skills Workshop, NeurIPS} \\ [3pt] 
James L. McClelland, Bruce L. McNaughton, and \textbf{Andrew K. Lampinen} (under review), {``Integration of new information in memory: new insights from a complementary learning sytems perspective''} \\[3pt]
\textbf{Andrew K. Lampinen} and Surya Ganguli, (2019), {``An analytic theory of generalization dynamics and transfer learning in deep linear networks,''} \textit{Proceedings of the 7th International Conference on Learning Representations} \\[3pt] 
\textbf{Andrew K. Lampinen} and James L. McClelland, (2018), {``Different presentations of a mathematical concept can support learning in complementary ways,''} \textit{Journal of Educational Psychology} \\[3pt]
 Robert X. D. Hawkins, Eric N. Smith, Carolyn Au, Juan Miguel Arias, Rhia Catapano, Eric Hermann, Martin Keil, \textbf{Andrew Lampinen}, Sarah Raposo, Jesse Reynolds, Shima Salehi, Justin Salloum, Jed Tan, and Michael C. Frank, (2018), {``Improving the replicability of Psychological Science through pedagogy,''}  \textit{Advances in Methods and Practices in Psychological Science} \\ [3pt]
Steven S. Hansen, \textbf{Andrew K. Lampinen}, Gaurav Suri, and James L. McClelland, (2017), {``Building on prior knowledge without building it in,''} \textit{Behavioral \& Brain Sciences}  \\[3pt]
\textbf{Andrew K. Lampinen}, Shaw Hsu, and James L. McClelland, (2017), {``Analogies emerge from learning dynamics in neural networks,''} \textit{Proceedings of the 39th Annual Meeting of the Cognitive Science Society}  

\vspace{1pt}\section{Preprints} \vspace{-15pt} \rule{\textwidth}{0.5pt} \\[3pt]
S\'ebastien Racani\`ere*, \textbf{Andrew K. Lampinen}*, Adam Santoro, David P. Reichert, Vlad Firoiu, and Timothy P. Lillicrap, (2019), {``Automated curricula through setter-solver interactions,''} \textit{arXiv}, (*equal contribution) \\ [3pt] 
Felix Hill, \textbf{Andrew K. Lampinen}, Rosalia Schneider, Stephen Clark, Matthew Botvinick, James L. McClelland, and Adam Santoro (2019), {``Emergent systematic generalization in a situated agent,''} \textit{arXiv} \\ [3pt] 
\textbf{Andrew K. Lampinen} and James L. McClelland, (2017), {``One-shot and few-shot learning of word embeddings,''} \textit{arXiv} \\ [3pt] 
\textbf{Andrew K. Lampinen}, David So, Douglas Eck, and Fred Bertsch, (2017), {``Improving image generative models with human interactions,''} \textit{arXiv} 

\vspace{1pt}\section{Invited Talks} \vspace{-15pt} \rule{\textwidth}{0.5pt} \\[3pt]
{``Multi-task learning, transfer, and abstraction,''} \textit{Parallel Distributed Processing and the Emergence of an Understanding of Mind}, Princeton University, September 29th, 2018\\[3pt] 
{``The Jabberwocky: One-shot and few-shot learning of word embeddings,''} \textit{Meaning in Context Workshop}, Center for the Study of Language and Information,  Stanford University, September 12th, 2017 
 
\vspace{1pt}\section{Presentations} \vspace{-15pt} \rule{\textwidth}{0.5pt} \\[3pt]
{``An analytic theory of generalization dynamics and transfer learning in deep linear networks,''} Natural / Artificial Intelligence, Stanford Neurosciences Institute, October 2018\\[3pt]
{``An analytic theory of generalization dynamics and transfer learning in deep linear networks,''} Parallel Distributed Processing and the Emergence of an Understanding of Mind, Princeton University, September 2018\\[3pt]
{``Analogies emerge from learning dynamics in neural networks,''} 39th Annual Meeting of the Cognitive Science Society, July 2017\\[3pt]
{``Fast and sparse learning with compositional concept training,''} 15th Neural Computation and Psychology Workshop, August 2016\\[3pt]
{``Cherenkov Radiation Based False Positive Detection for Rare Decays,''} Berkeley Undergraduate Physics Spring Poster Session, May 2012

\vspace{1pt}\section{Teaching Experience} \vspace{-15pt} \rule{\textwidth}{0.5pt} \\[3pt]
{\bf Teaching Assistant,} Stanford University Department of Psychology, 6 course between Fall 2016 and Winter 2019
\begin{itemize} \itemsep -2pt
  \item Planned and taught discussion sections for undergraduate statistics \& memory courses and graduate statistics \& research methods courses. \item Gave lectures on reinforcement learning and wrote and graded homeworks for graduate course on Neural Network Models of Cognition. \item Held office hours. \end{itemize}\vspace{-8pt}
{\bf Undergraduate Student Instructor,} UC Berkeley Mathematics, Spring, Fall 2013, \& Spring 2014 
\begin{itemize} \itemsep -2pt
  \item Planned and taught discussion sections. \item Held office hours. \item Wrote and graded quizzes and midterms. \end{itemize}\vspace{-8pt}
{\bf Teaching Assistant,} UC Berkeley Early Academic Outreach Program, June-July 2013
\begin{itemize} \itemsep -2pt
 \item Held office hours. \item Substitute taught classes. \end{itemize}

\vspace{1pt}\section{Other Work Experience} \vspace{-15pt} \rule{\textwidth}{0.5pt} \\[3pt]
{\bf Statistics Consultant,} Stanford University Department of Psychology, 2016-2017, 2019-2020
\begin{itemize} \itemsep -2pt
 \item Advised graduate students on technical aspects of data collection, analysis, and modeling. \end{itemize}
\vspace{1pt}\section{Service} \vspace{-15pt} \rule{\textwidth}{0.5pt} \\[3pt]
{\bf Reviewer:} 
\begin{itemize} \itemsep -2pt
 \item Journal of Educational Psychology.
 \item Cognitive Science Society, 2019.
 \item Conference on the Mathematical Theory of Deep Neural Networks (DeepMath), 2019\end{itemize}
\vspace{1pt}\section{Technical Skills} \vspace{-15pt} \rule{\textwidth}{0.5pt} \\[3pt]
{\bf Computer science:} Experienced with both theory and practice. 
\begin{itemize} \itemsep -2pt
  \item Graduate coursework in machine learning, neural networks, and probabilistic models \& algorithms.
  \item Experienced user of Python, R, C++, C, JavaScript, Matlab, some knowledge of Mathematica, Macaulay2, Haskell. 
  \item Used many common libraries for these languages, e.g. numpy, scipy, tidyr, dplyr, jquery, matplotlib, Matlab Computer Vision Toolbox, FFTW.
  \item Used many machine learning libraries, including TensorFlow, Torch, scikit-learn, and Caffe.
  \item Experienced with *NIX operating systems.
\end{itemize}\vspace{-8pt}
{\bf Mathematics:} Knowledge across many domains, with applications.
\begin{itemize} \itemsep -2pt
\item Algebraic geometry, group theory, category theory, topology, etc. \item Practical applications to machine learning, computer vision, neural coding, etc. \end{itemize}\vspace{-8pt}
{\bf Statistics:} Significant experience with standard data analysis techniques.
\begin{itemize} \itemsep -2pt
  \item Linear modeling, hierarchical modeling, etc.
  \item Fitting algorithms \& goodness-of-fit tests. \end{itemize} \vspace{-8pt}
{\bf Physics:} Experienced in a wide variety of applied and experimental contexts.\begin{itemize} \itemsep -2pt
\item Statistical mechanics, biophysics, analytic mechanics, etc. \item Experimentation ranging from NMR to quantum entanglement. \end{itemize}\vspace{-8pt}
{\bf Modeling \& Simulation:} Developed models and simulations of various phenomena. 
\begin{itemize} \itemsep -2pt
  \item Developed both from published methods and directly from physical principles. \end{itemize}
%{\bf Laboratory Equipment:} Competent with most common laboratory equipment. 
%\begin{itemize} \itemsep -2pt
% \item Oscilloscopes, standard \& lock-in amplifiers, signal generators, etc. \end{itemize} 
%%\vspace{-8pt}
%%{\bf Computer} Experienced with various operating systems and software applications.
%%\begin{itemize} \itemsep -2pt
%% \item Windows and Linux systems. \item Office, Photoshop \& Illustrator. \item  HTML \& \LaTeX.\end{itemize} 
\vspace{1pt}\section{Other Activities} \vspace{-15pt} \rule{\textwidth}{0.5pt} \\[3pt]
{\bf Carillon:} Carilloneur member of the Guild of Carilloneurs in North America (\url{www.gcna.org}). \\[3pt]
{\bf Rock climbing:} Bouldering, sport, and trad. Routesetter at Stanford Climbing Wall, set problems for Collegiate Climbing Series events.

\end{resume}
\end{document} 



